\documentclass{article}
\usepackage{setspace}
\usepackage{url}
\usepackage{geometry}
\geometry{a4paper,scale=0.7}

%++++++++++++++++++++++++TITLE++++++++++++++++++++++
\title{Socttish Engineer-William King}
\author {FENG Cheng-lin 2357707f\\}
\date{26 July 2018}

%++++++++++++++++++++++++SETTING+++++++++++++++++++++
\begin{document}
\maketitle
\begin{spacing}{2.0}
\linespread{2}

%+++++++++++++++++++++++BEGIN REPORT++++++++++++++++
\section{William Falconer King}
William Falconer King (17th April 1851- 6th October 1929), a scottish engineer, graduated as a postgraduate student of University of Glasgow. During his study in science department, he received the teaching of Sir William Thomson in electricity and mathematics.\\

\noindent Working as a chief engineer of the Western and Brazilian Telegraph Company, he was responsible for laying the Brazil's first submarine cable. Ten years later, he went back to Scotland and established King, Brown \& Co company with his friend Mr. Andrew Betts Brown who is a hydraulic engineer. One of the products of this company, the “300-light”, was actually used for illumination for the first Exhibition in Edinburgh in 1886.\\

\noindent At 7th June 1880, Mr. King was registered as a member of Royal Society of Edinburgh.\cite{bibitem1} He spent his twilight years at Hunter’s Quay and still keen on yachting as a member of the Yacht Clubs.

\section{Chinese Engineer}
YANG Wei, Chief designer of Chengdu J-20.


%++++++++++++++++++++BIBLIOGRAOHY+++++++++++++++++++++++
\begin{thebibliography}{10}
\bibitem{bibitem1}Waterston, C. and Shearer, A. (2006). Former fellows of The Royal Society of Edinburgh, 1783-2002. Edinburgh: The Royal Society of Edinburgh. 
\bibitem{bibtiem2}A. O.(1930). Proceedings of the Royal Society of Edinburgh. In: William F. King. vol. 49. Edinburgh: The Royal Society of Edinburgh, 1930, 386-387. 
\bibitem{bibtiem3}En.wikipedia.org. (2018). William King (engineer). [online] Available at: \url{https://en.wikipedia.org/wiki/William_King_%28engineer%29 [Accessed 25 Jul. 2018].} 
\end{thebibliography}

\end{spacing}
\end{document}