
\documentclass{article}
\usepackage{setspace}

	\title{The Linear Algebra Apllied in Neural Network}
	\author{CHEN Ming-yu; FENG Cheng-lin; YU Hong-ze}
	\date{May 6 2018}

\begin{document}
\maketitle
\begin{spacing}{2.0}

%%%%%%%%%%%%%%%%%%%%%%%%%%%%%%%%%%%%%%%%%%%%%%%%%%%%%%%%
%\begin{center}
\section{Introdunction of Neural Network}
%\end{center}

What is home? Such is a question that has left many people wondering for centuries. Now the concept of home \\


\section{Insert}
I insert a section here.




\section{Some math}\label{sec:section1}
\begin{math}
	$$\sum_{i=0}^{n}i^2$$\\
	$$\Delta$$\\
	$$\times$$
\end{math}\\

\section{woow. Anoter section}

I whant to refer to the math in section \ref{sec:section1}.
In J. W. Duyvendak’s article, although Duyvendak argues soundly against the traditional definition of home, he himself makes a similar mistake. He states that “the emotion of ‘feeling at home’ attracted less interest than the object of the feeling” (Duyvendak 42). In his words, people pay more attention to the object which contains the feeling of home than the feeling itself. The house is labeled as ‘home’, people will always name a physical place their home. It seems like only in that way, only if there is a specific space which they can name as home, can they then have a feeling of home. Duyvendak points out that people confine the abstract concept of home within the physical cage of house. Then, after experiencing the feeling at a house they claim that they find the feeling at home and use it to derive the meaning of home. Essentially, what thbey are doing is to first impose a boundary on the definition of home and then they try to seek a specific definition within the very boundary that they themselves created. In this sense, in order for people to find ‘home’ and its definition, they first have to open the cage and go beyond the boundary, that is, to leave ‘home’, the one that is previously defined by them. The process of finding home is like finding the missing pieces of a jigsaw puzzle. People start with a blank frame when they are born; then, they are given some pieces that constitute a portion of the whole puzzle. Also, these pieces would most likely be about the warmth and the positive feeling that a home can bring. In the section “Meanings of home” of “Why Feeling at Home Matters”, Duyvendak points out that “familiarity”, “haven”, and “heaven” are the three basic classifications that constitute the “elements of home” (Duyvendak 38). Familiarity refers to a sense of belonging; haven refers to a sense of security and comfort; heaven refers to a sense of exclusivity (Duyvendak 38). With no exception, the three classifications are all positive feelings that one’s home can bring; here, Duyvendak fails to recognize the fact that home can bring about negative feelings as well and these positive feelings, so is the mistake of many people. Many of them, like Duyvendak, just stop there, thinking that the small portion with positive feelings and the object of home that they see is the whole picture. Such is illustrated by the story of international students who left their home in “Home Away from Home: International Students and their Identity-Based Social Networks in Australia”.
\end{spacing}
\end{document}